% A few local macros that are used by the example content.
\newcommand{\expect}[2]{\mathds{E}_{{#1}} \left[ {#2} \right]}
\newcommand{\myvec}[1]{\boldsymbol{#1}}
\newcommand{\myvecsym}[1]{\boldsymbol{#1}}
\newcommand{\vx}{\myvec{x}}
\newcommand{\vy}{\myvec{y}}
\newcommand{\vz}{\myvec{z}}
\newcommand{\vtheta}{\myvecsym{\theta}}

\section{Introduction}

\kant[1]
\kant[2]
\kant[3]

\section{Using Figures}
%
We can add figures in the usual way. Figure \ref{fig:image1}.
\begin{figure}[t]
	\centering
	\includegraphics[width=\columnwidth]{kurt-cotoaga-1210012-unsplash}
	\caption{Image. This image comes from unsplash.com, which is a great website to get 
	free to use high quality images.}
	\label{fig:image1}
\end{figure}

\section{Latex Environments}
Using paragraph environment.
\paragraph{Opening Paragraph.} Paragraph is a way to have a bolded heading, and that can also 
enter into the pdf bookmark structure.

\section{Equations}
%
We can write equations this way:
\begin{align}
\log p(\vx) & = \log \int p_\theta(\vx,\vz) p(\vz) d\vz \nonumber \\
& = \log \expect{p(\vz)}{p_\theta(\vx,\vz)}
\label{eq:marginalisation1}
\end{align}
We refer to the previous equation \eqref{eq:marginalisation1}.
Later let's compute the gradient $\nabla_\theta \log p(\vx)$. The commands 
\verb|\vz|, \verb|\vx|, \verb|\expect| are locally-defined macros.
The file \texttt{defns.tex} provides a larger set of short macros for
common constructions, but some of them clash with existing packages.
\begin{align}
\log p(\vx) & = \nabla_{\vtheta} \sum_{i=1}^N \log p(y | x(\vtheta)) + \mathcal{R}(x) \nonumber \\
            & + \|\nabla_{\vtheta}\vx(\vtheta)\|^2_2 \\
            & y \in \mathbb{R}; \vx \in \mathbb{R}^D \qquad \text{using \texttt{\textbackslash mathbb}} \\
            & y \in \mathds{R}; \vx \in \mathds{R}^D \qquad \text{using \texttt{\textbackslash mathds}}
\label{eq:marginalisation2}
\end{align}

\subsection{Tables}
Use \href{https://www.tablesgenerator.com/latex_tables}{\texttt{www.tablesgenerator.com/latex\_tables}} to help make tables.

\begin{table}[tb]
	\centering
	\caption{Sizes of datasets. Testing with a much longer caption to see how it looks over 
	multiple lines. }
	\begin{tabular}{lll}
		\hline
		Dataset  & N      & D            \\
		\hline \hline
		MNIST    & 60,000 & $32\times32$ \\
		ImageNet & 1m     & $64\times64$\\
		\hline
	\end{tabular}
\end{table}

\subsubsection{Using lists}
%
Itemize lists
\begin{itemize}
	\item Item 1
    \item Item 2
	\item Item 3
\end{itemize}

\noindent Enumerate lists
\begin{enumerate}
	\item Item 1
	\item Item 2
	\item Item 3
\end{enumerate}

\section{DeepMind Brand Colours}
The brand standard specifies a colour palette that is available using the package \texttt{dm-colors}, which is already included in this template. Colours include: \textcolor{dmblue400}{This} \textcolor{dmyellow500}{text} \textcolor{dmteal400}{is} \textcolor{dmpurple400}{rendered} \textcolor{dmred400}{using} \textcolor{dmorange400}{dmcolors}.

\section{Including References and Bibliography}
\begin{figure*}[t]
	\centering
	\includegraphics[width=\columnwidth]{kurt-cotoaga-1210012-unsplash}
	\includegraphics[width=\columnwidth]{kurt-cotoaga-1210012-unsplash}
	\caption{Image. This image comes from unsplash.com, which is a great website to get 
		free to use high quality images.}
	\label{fig:image2}
\end{figure*}
References can be formatted in two styles with the \texttt{citep}
command \citep{silver2016mastering} and with the \texttt{citet}
command \citet{silver2016mastering}.
