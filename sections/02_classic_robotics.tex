\section{Robotics Foundations}
\label{sec:classical}

Robotics is concerned with producing real-world motion in a way that is useful, reliable and safe. 
At its core, robotics is thus an inherently multidisciplinar domain, as producing motion requires interfacing various hardware and software componets, each at various levels. 
As a direct consequence of its multi-disciplinar nature, robotics traditionally employed an array of methods concerned with its main purpose of producing motion.
Methods to produce robotics motion range from traditional \emph{explicit} models---leveraging precise descriptions of the mechanics of robots' rigid bodies and their interactions with eventual obstancles in the robots environment---to fully \emph{learning-based} systems, offloading modelling the robot mechanics and rather treating motion as a statistical pattern to learn given multiple sensorimotor readings.
Between these two extrema, a variety of methods exist, with many borrowing traits from the opposite extremum to tackle specific challenges. 
On the one hand, as learning-based systems can benefit from information relative to the physics of particularly static tasks, Temporal Difference (TD) methods have been complemented with Model-Predictive Control (MPC)~\citep{TDMPC}.
On the other hand, explicit models may be relying on assumptions proving overly simplicistic---or even unrealistic---in practice, as in the case of assuming perfect observability of a robot state using sensorimotor readings. In this context, feedback loops at the control level help mitigate the effects of poor state estimation, complementing the motion planning process with discrepancy-from-target information, similarily to how loss functions are used in statistical learning.
Figure~\ref{fig:generating-motion-atlas} graphically illustrates the most prominent models in general-purpose applications, and we refer the interested reader to~\citet{STATUS_OF_MOTION_GENERATION} for a comprehensive overview of both general and application-specific methods for motion generation.
Aiming at introducing the inherent benefit of learning-based approaches in the context of an increasing availability of robot-data---the main focus of this tutorial---in the rest of this section we will present an illustrative example for the task of manipulation using a rather simplicistic technique in the broader context of traditional robotics: inverse kinematics.
While far from more advanced techniques that have proven groundbreaking in specific real-world applications (among many others,~\citep{KATHIB_POTENTIAL_FIELD, SEQUENTIAL_COMPOSITION_OF_DYNAMICALLY_DEXTEROUS}), we believe the basics primitive of a full description of movement provide sufficient intuition as to the motivation of learning-based methods in the context of modern robotics.

\subsection{A Toy Example: A Planar Manipulator}
Locomotion, Manipulation, Whole body control: what are they

Toy example of planar robot

Full physical description by means of forward kinmeatics to generate movement

Most of the times you want to track something, so you do inverse kinematics

What is inverse kinematics

\subsubsection{Overcoming Estimation Error via Feedabck Loops}

Adding feedback loops to a (simplified) dynamical system

Tuning the gain is not immediate and rather cumbersome

\subsection{Limitations}

Reinforcement Learning for pivoting task's limitations for robot learning