\section{Foundation Robotics Models}
\label{sec:learning-bc-generalist}

\epigraph{\textit{Specialization is for insects}}{Robert A. Heinlein}

\subsection{Overview of Major Architectures: RT-1, RT-2, OpenVLA, $\pi_0$, $\pi_{0.5}$, SmolVLA}
VLA models integrate perception, instruction following, and action generation within a unified network. RT-1 demonstrated large-scale, robot-collected datasets and transformer policies capable of executing hundreds of distinct manipulation tasks; RT-2 extended this idea by co-training with web-scale vision–language corpora and representing actions as discrete tokens, improving semantic generalization to novel objects and instructions. OpenVLA adopts an open 7B backbone that fuses pretrained vision encoders with a language model and trains on large, diverse robot demonstrations, reporting strong cross-embodiment performance and efficient fine-tuning. The $\pi_0$ family introduces flow-matching action heads atop a pretrained VLM to produce continuous, high-frequency motor commands for dexterous skills; $\pi_{0.5}$ augments co-training for broader open-world generalization. Complementing these, SmolVLA targets efficiency and accessibility with a compact (\(\sim\!4.5\times10^8\) parameters) architecture and an interleaved attention ``action expert'' for low-latency control.

\subsection*{5.3\quad Practical Implementation: Integrating VLAs with \texttt{LeRobot}}
Integration typically follows three phases. \emph{Pretraining:} initialize the VLM backbone (e.g., SigLIP/DINOv2 features) and attach an action head (autoregressive, diffusion, or flow-matching) trained on heterogeneous demonstrations curated in a standardized format. \emph{Task adaptation:} fine-tune with instruction-augmented, robot-specific data using low-rank adaptation where possible; calibrate action scaling, control-rate chunking, and observation encodings. \emph{Deployment:} quantize or distill for on-robot inference; implement asynchronous inference stacks to decouple perception and control loops; add safety fallbacks and confidence gating. \texttt{LeRobot} dataset and model utilities simplify all three phases by providing consistent IO, evaluation tools, and model hosting.

\subsection*{5.4\quad Experimental Evaluation}
A principled evaluation protocol should measure (i) success rate across held-out tasks and objects, (ii) language grounding fidelity under paraphrases and compositional instructions, (iii) cross-embodiment transfer (train/test robot mismatch), and (iv) robustness to distribution shift (lighting, clutter, distractors). Where possible, report zero-shot and few-shot adaptation performance, ablate pretraining datasets and backbones, and include real-world trials with fixed seeds and video evidence. Public, diverse datasets and standardized success criteria are critical to comparable results.