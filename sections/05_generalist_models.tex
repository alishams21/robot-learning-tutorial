\section{Generalist Robotics Policies}
\label{sec:learning-bc-generalist}

\epigraph{\textit{Specialization is for insects}}{Robert A. Heinlein}

The advent of large models trained on internet-scale datasets has drastically influenced fields like Computer Vision (CV) and Natural Language Processing (NLP), with recipes consisting of (1) an initial, task-agnostic large-scale pre-training stage followed by a (2) task-specific adjustment of a base model emerging as the standard in many applications.
Such a paradigm largely replaced a more classic approach consisting of task-specific data collection, curation and model training, which inherently hindered cross-task scalability and is ultimately labor intensive, considering the impact both data curation and model training recipes have traditionally had on downstream performance.
Factors including (1) the advancements in generalist models for advanced semantic-aware perception~\citep{oquabDINOv2LearningRobust2024} and (2) popularization of widespread, decentralized efforts to collect large-scale openly available datasets~\citep{OpenXEmbodimentRobotic,DROIDLargeScaleIntheWild} are increasingly pushing the field of robot learning towards a similar paradigm, surpassing the traditionally limited interplay between tasks in robotics by scaling the size and diversity of accessible human demonstrations~\citet{brohanRT1RoboticsTransformer2023}.
While Section~\ref{sec:learning-bc-single} introduced BC methods for learning to perform tasks via \emph{single-task} policies such as ACT or Diffusion Policies, we here present advancements on developing \emph{robotics foundational models}. 

\subsection{Preliminaries: Large Datasets and Large Models}
% The successes of transformers in NLP

% The very small scale of robotics data compared to NLP datasets

% Early attempts:
- RT-1, RT-2
- Open-X and RT-X
- OpenVLA

- Problems with (1) casting robot control as visual question answering and (2) using discrete action spaces

\subsection{Using Action Experts}
- Introducing action experts conditioned on semantic features extracted by large scale models

\subsubsection{\( \pi_0 \)}
- The architecture of PI0
- Size of PI0
- Challenging for low resource applications

% \subsubsection{Train \( \pi_0 \)}
\subsubsection{Use \( \pi_0 \)}

\subsubsection{SmolVLA}
- A compact model which adopts many optimizations
- Modified action expert with CA/SA
- Better than specialist models on certain tasks

% \subsubsection{Train SmolVLA}
\subsubsection{Use SmolVLA}