\section{Conclusions}
\label{sec:conclusions}

This tutorial has charted the paradigmatic shift transforming robotics, tracing the \highlight{evolution of robotics from structured, model-based methods to the dynamic, data-driven approaches that define modern robot learning}. We began by examining the limitations of traditional dynamics-based control, namely its brittleness and significant engineering overhead, which motivate the adoption of more flexible, learning-based alternatives. Unlike scalable, data-driven techniques, conventional explicit models demand extensive human expertise, hindering wider accessibility and scalability of robotics.

Our exploration traced a clear trajectory of progress, beginning with Reinforcement Learning (RL). While RL offers a powerful paradigm for learning through interaction, its application in robotics is complicated by challenges such as sample inefficiency, safety concerns in real-world training, and the complexities of reward design. We saw how modern approaches like HIL-SERL make real-world RL more feasible by incorporating training-time human guidance, datasets of previously collected data as well as learned reward classifiers.

Nonetheless, the inherent difficulties of RL increasingly motivate approaches based on imitation learning, capable to safely learns from limited numbers of real-world, reward-free expert demonstrations. In turn, the wider adoption of imitation learning led to the development of single-task policies, where advanced Behavioral Cloning techniques---implemented as state-conditioned generative models like Action Chunking with Transformers and Diffusion Policy---have demonstrated the ability to learn complex, multimodal behaviors from human demonstrations. These advancements laid the groundwork for the current frontier: generalist, language-conditioned Vision-Language-Action models capable to perform few- and zero-shot a variety of different real-world tasks. By leveraging powerful pre-trained backbones and sophisticated generative methods like flow matching, models such as \pizero~and SmolVLA represent a significant leap towards foundational models for robotics capable of generalizing across diverse tasks, and even robot embodiments.

A central theme of this work is the critical role of openness in accelerating this progress. The recent explosion in capability is inseparable from the advent of large-scale, openly available datasets, standardized, stable and accessible model architectures, and accessible, open-source software like \lerobot. We argue this convergence on open-source robotics is not a mere trend but a fundamental enabler, democratizing access to research and unlocking the potential of large, decentralized efforts to advance the field.

The journey detailed in this tutorial, from first principles to the state-of-the-art, aims to equip researchers and practitioners with the context and tools to begin their own explorations in open-source robot learning.