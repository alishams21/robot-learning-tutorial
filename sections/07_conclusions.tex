\section{Conclusions}
\label{sec:conclusions}

This tutorial has chronicled the paradigmatic shift transforming robotics, from the structured, model-based methods of its classical era to the dynamic, data-driven approaches that define modern robot learning. 
We began by examining the limitations of traditional dynamics-based control, highlighting the brittleness and the significant engineering overhead required by traditional approaches, which in turn motivates more flexible, less model-intensive learning approaches.

Our exploration of learning-based techniques revealed a clear trajectory of progress. 
We began with Reinforcement Learning, acknowledging its power to learn through interaction but also its real-world challenges, particularly sample inefficiency and the complexities of reward design. 
We saw how modern, data-driven approaches like HIL-SERL can make real-world RL feasible by incorporating human guidance and prior data. 
The inherent difficulties of RL, however, naturally motivated a deeper dive into imitation learning. This led us to single-task policies, where Behavioral Cloning, powered by advanced generative models like Action Chunking with Transformers and Diffusion Policy, demonstrated the ability to learn complex, multimodal behaviors directly from expert demonstrations. 
This laid the groundwork for the current frontier: the development of generalist, language-conditioned Vision-Language-Action models. 
Architectures like \( \pi_0 \) and SmolVLA---leveraging powerful pre-trained backbones and sophisticated generative modeling techniques like flow matching---represent a significant leap towards building foundational models for robotics that can generalize across varied tasks and embodiments.

A central theme throughout this work has been the critical role of openness in accelerating this progress. 
The recent explosion in capability is inseparable from the advent of large-scale, openly available datasets, the standardization of powerful and efficient model architectures, and the development of accessible, open-source software like \lerobot. 
We argue the convergence towards an open approach to robotics is not merely a trend but a fundamental enabler, democratizing access to cutting-edge research in a traditionally siloed field like robotics.

We believe the path ahead for robot learning to be overly exciting, and filled with fundamental challenges we yet have to even scratch the surface of.
The journey detailed in this tutorial, from the first principles to the state-of-the-art, equips researchers and practitioners alike with the context and the tools to chart their own journey in the future of robotics.